\chapter{Introduction}
Electricity price forecasting attempts to deduce in advance what the price of electricity will be.
This is a difficult task, as many factors interfere with it: raw materials (as natural gas, coal or uranium) prices, weather conditions or energy demand.
Apart, the local and global economic situation or tax policies and regulations approved by the government affect, among others.

In this Master's Thesis we want to perform forecasts of the energy price in the short, medium and long term, studying which variables affect the price in the Spanish market.
Concretely, we want to study hourly, daily, monthly and yearly aggregations.
Today, after the covid crisis of previous years and the war in Ukraine, prices have been greatly altered, reaching values that have never been seen before.
In this situation, the completion of this Thesis makes a lot of sense, as we may discover how the market has changed after these two events.

To achieve this goal the author will first select the variables needed to perform the study, download historical data related with them and apply time series analytics tools, both statistical and machine learning based, to understand the data and make forecasts.

\section{Project motivation}


\section{Objectives}
The project has the following goals:
\begin{itemize}
    \item \textbf{[OBJ1]} Understand which variables affect the electricity price in the short, medium and long term.
    \item \textbf{[OBJ2]} Forecast electricity price in the short, medium and long term.
\end{itemize}