\chapter{Introduction}
Electricity price forecasting attempts to deduce in advance what the price of electricity will be.
This is a really difficult task, as many factors interfere with it: raw materials (as natural gas, coal or uranium) prices, weather conditions or energy demand.
Apart, the local and global economic situation or tax policies and regulations approved by the government affect, among others.

In this Master's Thesis we want to perform forecasts of the energy price in the short, medium and long term, studying which variables affect the price in the Spanish market.
Today, after the covid crisis of previous years and the war in Ukraine, prices have been greatly altered, reaching values that have never been seen before.
In this situation, the completion of this Thesis makes a lot of sense, as we may discover how the market has changed after these two events.


\section{The Spanish electricity market}
In this section we briefly describe how the electricity market works in Spain.
In 1998, the current regulation came into operation, replacing the previous system in which Red Eléctrica de España (REE) and the Ministry of Industry and Energy entirely planned their transactions.

It allowed the private sector to freely plan the construction of generation plants and the possibility of bidding energy prices, and created the figure of energy supplier (and the freedom to choose supplier for the consumer).
Distribution of energy remained regulated by the government, who assigns this task to a unique company per region. \cite{mercado-electrico-mincotur}

\subsection{Main agents}
Different agents conform this complex system. Each one is in charge of a specific task:\cite{mercado-electrico-endesa, organismos-reguladores-holaluz}

\begin{itemize}
    \item \textbf{Generators} produce energy and should build, operate and maintain the power plants.
    \item \textbf{Transporters} take the energy from the power plants to the distribution network, building and maintaining the transport network. This network operates in high voltage, making transmission through long distances more efficient.
    \item \textbf{Distributors} extract the energy from the transport network and supply it to the final consumers. They are in charge of building and maintaining the distribution network.
    \item \textbf{Suppliers} buy the energy to generators and sell it to final consumers.
    \item \textbf{Regulators} are in charge of legislating (government administration) and ensuring effective competition (Comisión Nacional de los Mercados y la Competencia, CNMC) in the energy market.
    \item There exist two \textbf{operators}, the market operator and the system operator. The former (Operador de Mercado Ibérico de Energía, OMIE) manages the day-ahead and intraday electricity markets. The latter (Red Eléctrica de España, REE) ensures the system stability, checking that generation and consumption are balanced every time as energy can't be stored in large scale, preventing lack of supply or overload on the power grid. Both operators should work closely, so they can face adequately to problems in the system.
\end{itemize}

\subsection{Main electricity generation technologies}


\section{The electricity price and the main variables defining its value}
El precio mayorista vs el precio final de la energia
Mercado diario e intradiario


Even though many variables affect those prices