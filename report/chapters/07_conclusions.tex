\chapter{Conclusions}
In this Master Thesis, the author has forecasted electricity prices and analyzed which variables are influencing the prediction. This analysis has been done in an hourly, daily (short term), monthly (medium term) and yearly  (long term) fashion, as each aggregation has different properties. This project has not only helped the author to begin to understand the electricity market, but to learn how to work with time series.

About the results obtained in the forecasts it can be seen how tree based algorithms performed the best, compared with distance based ones. Apart, modeling the short term has been easier than forecasting the medium, probably due to the higher number of datapoints in the former or the patterns in the series that make it more modelable. The MASEs obtained in the medium term are higher and the prediction intervals are broader, compared with those in the short term forecasts.

The results show that the probably most important predictor in the short term forecast is combined cycle generation: when predicting in an hourly fashion, it's the one that always appears in the top, together with some lags. In the daily aggregation this predictor still appears, but others related with renewable technologies such as wind power or hydropower also are important, apart from coal generation. The use of wind power seems to reduce electricity price, while the others contribute to increase it.

About pre-covid and post-war markets, in hourly and daily aggregations, the MASE obtained in post-war is always higher. This is probably due to the higher uncertainty on this period. About predictor importance, the tendencies in both periods are similar.

The yearly analysis has been done not analytically but in a descriptive way, due to the lack of data. On it, commodity prices seem to be the main driver of electricity price. The increase in solar and wind power generation could help to reduce their influence in the following years.

Finally, by conducting this Thesis, the author has discovered the state of the art in time series. He has learned how to work with the main libraries in Python, downloading and managing data retrieved from an API.